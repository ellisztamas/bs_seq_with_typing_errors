%
% Paper title
% Authors
% 
%-----------------------------------------------------------------

\documentclass[12pt,longbibliography]{article}

% Packages
%-----------------------------------------------------------------

% Allow direct use of accents such as á é ñ.
\usepackage[utf8]{inputenc}

% This is a good idea to have some symbols included within the font
% properly:
% http://tex.stackexchange.com/questions/664/why-should-i-use-usepackaget1fontenc
\usepackage[T1]{fontenc}

% Set type of paper and margins.
\usepackage[a4paper, margin=2.7cm]{geometry}

\usepackage{amsmath, amsthm, amsfonts, amssymb}
\usepackage{mathrsfs}           % \mathscr font.

% Generate a PDF with hyperlinks in references.
\usepackage[colorlinks=true,linkcolor=blue,citecolor=blue,urlcolor=blue,breaklinks]{hyperref}

% Double-stroke font (\mathbbm).
\usepackage{bbm} 


% Bibliography
%-----------------------------------------------------------------

% This uses a bibliography style which hyperlinks the paper titles to
% the paper URL specified in the bibtex file. It also uses natbib,
% which cites papers by name such as Euler (1770) instead of [17].

\usepackage{breakurl}
\usepackage{natbib}
\usepackage{url}
\bibliographystyle{plainnat-linked}
% \bibliographystyle{plain}

% Title, author, date
%-----------------------------------------------------------------

% If set, these will be the internal title and author of the PDF (and
% will be listed for example in ereaders and tablets).

% \hypersetup{pdftitle={Title of the PDF}}
% \hypersetup{pdfauthor={Author of the PDF}}


% Paper title and author
\title{Bisulphite sequencing in the presence of cytosine-conversion errors} % Article title

\author{
    Thomas James Ellis 
    \and
    Rahul Pisupati 
    \and
    Gr\'{e}goire Bohl-Viallefond
    \and
    Almudena Moll\'a Morales
    \and
    Magnus Nordborg
}



% Date is set automatically unless specified.
% \date{October 2015}

\date{\today} % Leave empty to omit a date

\begin{document}
\maketitle

\begin{abstract}
    Cytosine methylation is a common epigenetic mark, and is associated with silencing of transposable elements.
    Bisulphite treatment of DNA, leading to the conversion of unmethylated cytosines to thymine, is a common approach to infer the methylation status of cytosines.
    'Tagmentation' approaches to bisulphite treatment use a transposase to simultaneously make double-stranded breaks and ligate adaptors to the resulting fragments.
    This facilitates higher throughput of samples than is practical using traditional protocols that rely on sonication.
    However, it has also been noted that certain tagmentation protocols have an unusually high number unmethylated cytosines that are not converted to thymine.
    Here we describe this phenomenon in detail, and find that this is unlikely to be due to PCR or bioinformatic artefacts.
    We tentatively suggest that the issue is due to single strand nicks by the transposase.
    Nevertheless we show that these errors can be accounted for in downstream analysis, and that for many applications they are sufficiently small not to affect biological conclusions.
    \end{abstract}



% \author[\centering AUTHOR]{
%     \so{Thomas James Ellis, Viktoria Nyzhynska, Rahul Pisupati, Gr\'{e}goire Bohl-Viallefond, Almudena Moll\`a Morales, Magnus Nordborg}
%     \thanks{Corresponding author.\hfil\break e-mail: magnus.nordborg@gmi.oeaw.ac.at}}
%     \affiliation{Gregor Mendel Institute of Molecular Plant Biology, Doktor-Bohr-Gasse 3, 1030 Vienna, Austria}


% \history{Manuscript received xx xxxx xx; in final form xx
% xxxx xx}

\maketitle

% \begin{keywords}
% methylation, bisulphite, transposase, tagmentation
% \end{keywords}

\section{Introduction}

Bisulphite 
Here's how it works

Old protocols use such and such
They don't scale well
Transposase based assays can be scaled up because the transposase cuts and ligates at the same time.
Idea: digest, gap repair, ligate adaptors

Standard protocol uses unmethylated dNTPs, but it's finicky
Other protocol uses methylated dNTPs, which is better because reasons.
However, two authors noticed that this causes non-conversion errors.
They used ad hoc methods to correct for that,
We lack a detailled understanding of these errors and how to deal with them.

What could explain this?
Reads not being trimmed correctly, or poor conversion.
We address this, but it's unlikely because these are the same between protocols
More likely to be to do with the strand displacement/gap repair step, which differ between protocols
dNTPS with mC are being incorporated, for example via repair of single-strand nicks.
PCR bias?

\section{Results}

\subsection{Desrcibe the problem}

fig 1:
    A: 3 Bar plots for conversion rate, with PCR cycles on x axis, colours for concentration transposase
    B: errors across the read for one sample of each
    C: Distribution of error counts for one sample of each
    SI: give all the examples of B and C for all the samples I have

We get the problem in diverse organisms (plant, flies, virus)
    Plot overall non-conversion rates across samples, coloured by PCR cycles and concentration
Reads fall into three categories:
    Most reads are not affected
    some proportion have some methylation
    some have full methylation
Errors across reads:
    Sequence quality is high across the reads
    Errors more likely towards the 3` end.
    Sometimes a particular strand, often the complementary strands, look worse, but not always
        Stochasticity at initial PCR
    There's some whacky oscillation shit happening with drosophila. I've no idea why.

\subsection{Non-conversion errors vary across the genome}

\begin{figure}
    fig2:
        A: Panel with four rows showing mC across chloroplast for
            - technical reps
            - biological reps for 6191
            - biological reps for 6046
            - A sample of F2s
        B: Compare variance to binomial simulations
        C: grouped box plot showing non-conversion over annotated features for Col-0
        D: correlation of non-conversion rate and GC content
\end{figure}

(Need to make sure I'm comparing like with like - same window size)

Errors vary across the chloroplast
This variation is consistent between technical replicates
    window explains X percent of the variance
Fairly consistent across biological replicates
Fairly consistent across F2s
    windows explain 32\% of variance
Seems there is something about the structure of the genome that affects error rate

There's more variation than can be accounted for by binomial sampling.
The distribution looks Beta.
This can't be explained by functional annotations.
Check GC content

Conclusion: error rate seems to vary over DNA in a consistent, if not easily explainable way.
This means true methylation level for a given is uncertain, and a single point estimate may not always be appropriate.

\subsection{Methylation estimates depend more on sample size than non-conversion errors}

fig:
    deterministic error rates
    simulation, showing data without errors, data with known errors, data with best guess, estimate accounting for uncertainty
    Plot variance for ibid as a function of coverage


Estimating mean methylation is a binomial process
We can account for errors by allowing observed state of methylated and unmethylated reads to be correct or incorrect.
We model the probability that (1) truly unmethylated cytosines appear methylated (false non-conversion) or (2) truly methylated cytosines appear unmethylated (false converion; see Materials and Methods for full details).
These sources of error have the strongest effects when mean methylation is close to zero and one respectively (figure \ref{fig:}).
The true mean methylation level accounting for these error rates can then be estimated algebraicly (see Materials and Methods).
Can also estimate it accounting for uncertainty in theta.

To investigate how this works we simulated data.
We simulated loci of 250 cytosines, corresponding to TE's of 1000 bp, because the mean is 798.
Mean methylation of 0.15, because this is realistic for TEs and fairly close to the error distribution.
At each locus we drew and error rate from a Beta
We also summed loci
This gives biologically realistic data

When error rates are known we can correct for them as well as if there were no errors.
When errors are not know there is more error
May or may not be more when we integrate.
However, binomial sampling is huge
For small samples variance between estimates is massive even for perfect data
Statement about change in variance.
Summing over loci works
You need to some over about 100 TEs
As long as you sum enough, conversion errors don't matter.

\subsection{Reliable imputation of methylation state}

Fig: Three panels showing calls on chloroplast, TEs and genes.

Differences in methylation are often due to the effects distinct biological pathways
For example, Zhang et al split genes into unmethylated 
In this case it makes more sense to describe the state
For example, Zhang et al. distinguish unmethylated, CG- and TE-like methylation

Gaut and pals do something, but it's not ideal
We can do better by modelling the probability that methylation in each context
comes from the error process, or from an alternative process with more methylation
than that.



\section{Discussion}

\section{Materials and Methods}

\subsection{Biological material}

\subsection{Statistical models to accounts for conversion errors}

\subsubsection{Error rates in a binomial model}

Methylation pipelines tell us that of a total of $n$ reads mapping to a region of a genome, we observe $y$ reads mapping to methylated cytosines and $n-y$ reads mapping to unmethylated cytosines. The goal is to estimate the true mean methylation level $\theta$ which generated these data, accounting for conversion errors.

In the absence of errors, the likelihood of the data given $\theta$ is binomially distributed as
\begin{equation}
    \label{eqn:classic-binomial}
    \Pr(y| \theta) = {n \choose y} \theta^y(1-\theta)^{n-y}
\end{equation}
with mean $\theta=y/n$.
Data are not perfect, so we would like to incorporate two error terms:
\begin{itemize}
    \item $\lambda_1$ is the probability that an unmethylated cytosine appears methylated (the bisulphite non-conversion rate).
    \item $\lambda_2$ is the probability that a methylated cytosine appears unmethylated.
\end{itemize}
Cytosines observed to be methylated may thus be either truly methylated with probability $\theta(1-\lambda_2)$ or truly unmethylated with probability $(1-\theta)\lambda_1$. Likewise, a cytosine observed to be unmethylated may be either truly unmethylated with probability $(1-\theta)(1-\lambda_1)$ or truly methylated with probability $\theta \lambda_2$.
This changes the likelihood to
\begin{equation}
    \label{eqn:binom-with-errors}
    \Pr(y | \theta, \lambda_1, \lambda_2) = 
    {n \choose y}
    [\theta(1-\lambda_2) + (1-\theta)\lambda_1]^y
    [\theta \lambda_2 + (1-\theta)(1-\lambda_1)]^{n-y}
\end{equation}
Summarising $p=[\theta(1-\lambda_2) + (1-\theta)\lambda_1]$ for brevity, this has a closed-form maximum-likelihood estimate
\begin{equation}
    \label{eqn:ml-theta}
    \hat{\theta} = \frac{\lambda_1-p}{\lambda_1 + \lambda_2 -1}
\end{equation}

\subsubsection{Uncertainty in error rates}

The above formulation is valid for the case where there is a clear point estimate for error rates.
If there is substantial uncertainty around these estimates values, and especially if they are likely to vary across the genomes, then the $\theta$ and error rates can be modelled as coming from Beta distributions with shape parameters $a$ and $b$:
\begin{equation}
    \theta \sim \textrm{Beta}(a_{\theta}, b_{\theta})
\end{equation}
\begin{equation}
    \lambda_1 \sim \textrm{Beta}(a_{\lambda_1}, b_{\lambda_1})
\end{equation}
\begin{equation}
    \lambda_2 \sim \textrm{Beta}(a_{\lambda_2}, b_{\lambda_2})
\end{equation}
The distributions over $\Pr(\theta)$, $\Pr(\lambda_1)$ and $\Pr(\lambda_2)$ represent prior distributions of $\theta$ and error rates, and can be estimated from other data.
For example, $\lambda_1$ can be estimated by counting the proportion of all unconverted cytosines in windows of DNA that is known to be unmethylated to get a distribution of error rates, and taking mean $\bar{x}$ and variance $\sigma^2_x$ of these estimates.
The shape parameters of the Beta distribution for $\Pr(\lambda_1)$ can be calculated by method-of-moments as
\begin{equation}
    \hat{a}_{\lambda_1} = \bar{x}(\frac{\bar{x}(1-\bar{x})}{\sigma^2_x}-1)
\end{equation}
\begin{equation}
    \hat{b}_{\lambda_1} = 1-\bar{x}(\frac{\bar{x}(1-\bar{x})}{\sigma^2_x}-1) 
\end{equation}

We can use this information to estimate $\theta$ by integrating out values for error rates.
The joint posterior distribution of $\theta, \lambda_1, \lambda_2$ is then
\begin{equation}
    \Pr(\theta, \lambda_1, \lambda_2 | y, \pi)
    \propto 
    \Pr(y | \theta, \lambda_1, \lambda_2)
    \Pr(\theta | a_{\theta}, b_{\theta})
    \Pr(\lambda_1 | a_{\lambda_1}, b_{\lambda_1})
    \Pr(\lambda_2 | a_{\lambda_2}, b_{\lambda_2})
\end{equation}
where $\pi=\{a_{\theta}, a_{\lambda_1}, a_{\lambda_2}, b_{\theta}, b_{\lambda_1}, b_{\lambda_2}\}$ for brevity.
The marginal posterior distribution for $\theta$ can be found by integrating over $\Pr(\lambda_1)$ and $\Pr(\lambda_2)$.
The posterior mean of this distribution is the most probable value of $\theta$ accounting for all possible values of error rates. It ought to be possible to get a closed-form point estimate by setting the first derivative to zero and solving for $\theta$, but I haven't work out how yet.

\subsubsection{Methylation status}

In some cases it may be advantageous to think of DNA methylation as being in particular discrete states.
For example, [Zhang et al. (2020)](https://www.pnas.org/doi/full/10.1073/pnas.1918172117) present a case that DNA methylation (at least on genes) represents a continuum between unmethylated DNA, TE-like methylation (CG, CHG and CHH methylation), and CG-only methylation.
In this case it is likely that these states represent the effects of distinct biological pathways, so makes more sense to identify discrete states than to make quantitative estimates of methylation.
We would thus like to partition the genome into regions that look like each of these contexts.

Given total numbers of apparently methylated and unmethylated reads at each site ($y_{CG}$, $y_{CHG}$ and $y_{CHH}$) and reasonable point estimates of $\lambda_1$ and $\lambda_2$ we can use the estimate of $\hat{\theta}$ from \ref{eqn:ml-theta} above to calculate likelihoods that a window is unmethylated ($L_U$), TE-like methylated ($L_{TEm}$), or CG-methylated ($L_{CG}$):
\begin{equation}
    \label{eqn:lik-unmethylated}
    L_{U} = 
    \Pr( y_{CG} | \theta_{CG} =0, \lambda_1, \lambda_2)
    \Pr(y_{CHG} | \theta_{CHG}=0, \lambda_1, \lambda_2)
    \Pr(y_{CHH} | \theta_{CHH}=0, \lambda_1, \lambda_2)
\end{equation}
\begin{equation}
    \label{eqn:lik-mCG}
    L_{CG} = 
    \Pr( y_{CG} | \theta_{CG}>0, \lambda_1, \lambda_2)
    \Pr(y_{CHG} | \theta_{CHG}=0, \lambda_1, \lambda_2)
    \Pr(y_{CHH} | \theta_{CHH}=0, \lambda_1, \lambda_2)
\end{equation}
\begin{equation}
    \label{eqn:lik-TEm}
    L_{TEm} = 
    \Pr( y_{CG} | \theta_{CG}>0, \lambda_1, \lambda_2)
    \Pr(y_{CHG} | \theta_{CHG}>0, \lambda_1, \lambda_2)
    \Pr(y_{CHH} | \theta_{CHH}>0, \lambda_1, \lambda_2)
\end{equation}

This assumes errors are the same for CG, CHG and CHH sites; this is reasonable because we think errors happen in a tube, not a cell. It is probably sensible to set a minimum bound on the number of CG sites in a window such that you could actually expect to distinguish $\theta_{CG}$ and error rates.

One could just take the model with the highest likelihood (better: AIC), and that may well be enough for our purposes. To perform formal likelihood ratio tests we first note that $L_{CG}$ is nested within $L_{TEm}$ and $L_{U}$ is nested within $L_{CG}$. Likelihood ratio tests can then be performed assuming the test statistic under the null is $\chi^2$-distributed 
with 1 degree of freedom.



We can likewise modify the likelihood calculations of methylation statuses to include distributions of error rates.
In this case we define distinct prior distributions for the expected distribution of true methylation under different scenarios, and calculate the support for each.
For example, it is reasonable to expect that unmethylated DNA has a true methylation rate very close to zero, and hence we might define parameters for $\Pr(\theta)$ that give values very small means.

\end{document}